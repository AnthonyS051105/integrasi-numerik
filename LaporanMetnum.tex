\documentclass[a4paper, 12pt]{article}

%------------------------------------------------%
% PENGATURAN PAKET DAN DOKUMEN
%------------------------------------------------%
\usepackage[utf8]{inputenc}
\usepackage[bahasa]{babel}      % Mengatur bahasa Indonesia
\usepackage{graphicx}           % Untuk memasukkan gambar (logo)
\usepackage{amsmath}            % Untuk simbol dan lingkungan matematika
\usepackage{geometry}           % Untuk mengatur margin halaman
\usepackage{times}              % Menggunakan font Times New Roman

% Pengaturan margin halaman
\geometry{
    a4paper,
    left=25mm,
    right=25mm,
    top=25mm,
    bottom=25mm,
}

% Pengaturan jarak antar paragraf (non-indented)
\usepackage{parskip}
\setlength{\parindent}{0pt}
\setlength{\parskip}{1.5ex}

% Menonaktifkan nomor halaman untuk halaman pertama
\thispagestyle{empty}

%------------------------------------------------%
% AWAL DOKUMEN
%------------------------------------------------%
\begin{document}

%------------------------------------------------%
% BAGIAN JUDUL (COVER)
%------------------------------------------------%
\begin{center}
    % Logo UGM
    \includegraphics[width=3cm]{HitamPutih.jpg} % Pastikan file logo ada di folder yang sama

    \vspace{0.1cm} % Memberi jarak vertikal

    % Informasi Universitas
    \textbf{UNIVERSITAS GADJAH MADA} \\
    \textbf{FAKULTAS TEKNIK} \\
    \textbf{DEPARTEMEN TEKNIK ELEKTRO DAN TEKNOLOGI INFORMASI}

    \vspace{0.1cm}

    % Garis horizontal pemisah
    \rule{\textwidth}{1.2pt}

    \vspace{0.2cm}

    % Judul dan Subjudul
    \textbf{TUGAS METODE NUMERIS} \\
    \vspace{0.1cm}
    \textbf{\Large INTEGRASI NUMERIK DENGAN METODE \\ TRAPEZOIDAL, ROMBERG, ADAPTIVE, DAN GAUSSIAN QUADRATURE} \\

    \vspace{0.25cm}

    % Informasi Penulis
    Ditulis oleh: \\
    \vspace{0.1cm}
    \textbf{\large Kelompok:} \\
    \vspace{0.1cm}
    \textbf{1. Nathanael Satya Saputra (NIM NAEL)} \\
    \textbf{2. Muhammad Nafal Zakin Rustanto (24/535255/TK/59364)} \\
    \textbf{3. Yohanes Anthony Saputra (NIM ANTHONY)} \\
    \textbf{4. Johannes De Deo Dimas Aryobimo (NIM BIMO)}
\end{center}

\vspace{0.5cm} % Jarak pemisah sebelum masuk ke konten laporan

%------------------------------------------------%
% BAGIAN 1: TEORI TIAP METODE
%------------------------------------------------%
\section*{BAGIAN 1: Dasar Teori}

\subsection*{1.1. Integrasi Analitik (Metode Eksak)}

Integrasi analitik adalah metode perhitungan integral menggunakan rumus-rumus kalkulus secara langsung. Metode ini memberikan nilai eksak (tepat) dari suatu integral jika fungsi yang diintegralkan memiliki antiturunan yang dapat ditentukan.

\textbf{Definisi Integral Tentu:}

Integral tentu dari fungsi $f(x)$ pada interval $[a, b]$ didefinisikan sebagai:
\[
I = \int_{a}^{b} f(x) \, dx = F(b) - F(a)
\]
di mana $F(x)$ adalah antiturunan dari $f(x)$, yaitu $F'(x) = f(x)$.

\textbf{Teorema Fundamental Kalkulus:}

Jika $f(x)$ kontinu pada interval $[a, b]$ dan $F(x)$ adalah antiturunan dari $f(x)$, maka:
\[
\int_{a}^{b} f(x) \, dx = F(x) \Big|_{a}^{b} = F(b) - F(a)
\]

\textbf{Kegunaan:}

Hasil dari integrasi analitik digunakan sebagai nilai pembanding (nilai eksak) untuk mengevaluasi akurasi metode-metode integrasi numerik. Error dari metode numerik dihitung sebagai selisih absolut antara hasil numerik dengan nilai eksak ini.

\subsection*{1.2. Metode Trapezoidal Rule}

[Isi teori tentang metode Trapezoidal Rule]

\subsection*{1.3. Metode Romberg Integration}

[Isi teori tentang metode Romberg Integration]

\subsection*{1.4. Metode Adaptive Integration}

[Isi teori tentang metode Adaptive Integration]

\subsection*{1.5. Metode Gaussian Quadrature}

[Isi teori tentang metode Gaussian Quadrature]

%------------------------------------------------%
% BAGIAN 2: LANGKAH PERHITUNGAN
%------------------------------------------------%
\section*{BAGIAN 2: Langkah Perhitungan}

\subsection*{2.1. Perhitungan Integrasi Analitik (Nilai Eksak)}

\textbf{Fungsi 1: } $f(x) = \cos(x)$ pada interval $[0, \frac{\pi}{2}]$

Langkah perhitungan:
\begin{align*}
I_1 &= \int_{0}^{\pi/2} \cos(x) \, dx \\
&= \sin(x) \Big|_{0}^{\pi/2} \\
&= \sin\left(\frac{\pi}{2}\right) - \sin(0) \\
&= 1 - 0 \\
&= 1
\end{align*}

\textbf{Nilai Eksak: } $I_1 = 1.0$

\vspace{0.3cm}

\textbf{Fungsi 2: } $f(x) = x^2$ pada interval $[0, 1]$

Langkah perhitungan:
\begin{align*}
I_2 &= \int_{0}^{1} x^2 \, dx \\
&= \frac{x^3}{3} \Big|_{0}^{1} \\
&= \frac{1^3}{3} - \frac{0^3}{3} \\
&= \frac{1}{3} - 0 \\
&= \frac{1}{3}
\end{align*}

\textbf{Nilai Eksak: } $I_2 = 0.333333...$ atau $\frac{1}{3}$

\subsection*{2.2. Perhitungan Metode Trapezoidal Rule}

[Isi langkah-langkah perhitungan menggunakan metode Trapezoidal Rule]

\subsection*{2.3. Perhitungan Metode Romberg Integration}

[Isi langkah-langkah perhitungan menggunakan metode Romberg Integration]

\subsection*{2.4. Perhitungan Metode Adaptive Integration}

[Isi langkah-langkah perhitungan menggunakan metode Adaptive Integration]

\subsection*{2.5. Perhitungan Metode Gaussian Quadrature}

[Isi langkah-langkah perhitungan menggunakan metode Gaussian Quadrature]

%------------------------------------------------%
% BAGIAN 3: HASIL DAN PERBANDINGAN ERROR
%------------------------------------------------%
\section*{BAGIAN 3: Hasil dan Perbandingan Error}

\subsection*{3.1. Hasil Perhitungan}

[Isi tabel hasil perhitungan dari semua metode]

\subsection*{3.2. Perbandingan Error}

[Isi analisis perbandingan error dari semua metode]

\subsection*{3.3. Kesimpulan}

[Isi kesimpulan dari hasil dan perbandingan error]

%------------------------------------------------%
% AKHIR DOKUMEN
%------------------------------------------------%
\end{document}